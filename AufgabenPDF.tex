
\documentclass[paper = a4, ngerman]{scrartcl}

\usepackage[super,sort&compress]{natbib} 

\usepackage{varioref}

\usepackage{hyperref}

\usepackage[T1]{fontenc}

\usepackage{makeidx}
\makeindex

\usepackage{listings}
\usepackage{color}

\definecolor{dkgreen}{rgb}{0,0.6,0}
\definecolor{gray}{rgb}{0.5,0.5,0.5}
\definecolor{mauve}{rgb}{0.58,0,0.82}
\definecolor{light-gray}{gray}{0.95}
\lstset{frame=single,
	language=Java,
	aboveskip=3mm,
	belowskip=3mm,
	showstringspaces=false,
	columns=flexible,
	basicstyle={\small\ttfamily},
	numbers=none,
	backgroundcolor=\color{light-gray},
	numberstyle=\tiny\color{gray},
	keywordstyle=\color{blue},
	commentstyle=\color{dkgreen},
	stringstyle=\color{mauve},
	breaklines=true,
	breakatwhitespace=true,
	tabsize=3
}
\usepackage[miktex]{gnuplottex}

\usepackage{amsmath,amssymb}

\usepackage[noabbrev]{cleveref}

\usepackage{amsfonts}

\setlength{\parindent}{0pt} %Nicht Einrücken

\usepackage{color}


\usepackage[ngerman]{babel}

\usepackage[utf 8]{inputenc}

\usepackage{float}

\usepackage{graphicx}

\usepackage{alltt}


\subject{Diskrete Strukturen}
\title{Blatt [NR]}
\author{Eike Janning, Tobias Nagel, Jan Paul Wessendorf}
\date{Abgabe: [TAG].[MONAT].2019}

\renewcommand{\arraystretch}{1.25}



\begin{document}
\maketitle
\hrulefill


\section*{Aufgabe 13:}

Menge M, $|M| = n, R, R_1, R_2 $ Relationen auf M\\

\begin{itemize}
	\item[a)]
		$\rightarrow |R| \ge n \Rightarrow $ R reflexiv\\
		$M := \{a,b\} \Rightarrow |M| = n = 2$\\
		$M \times M = \{(a,a),(a,b),(b,a),(b,b)\}$\\
		Da $|R| \ge n$ folgt, da $n = 2$, dass $|R| \ge 2$\\
		Da $R \subseteq M \times M$ gilt z.B.: $R := \{(a,b),(b,a)\}$\\
		Damit ist die Reflexivität nicht erfüllt, da $\forall m \in M : (m,m) \in R$.\\
	\item[b)]
		$R_1 \subseteq R_2$
		\begin{itemize}
			\item[i)]
				$R_1$ reflexiv $\Rightarrow R_2 $ reflexiv\\
				$\rightarrow \forall m \in M : (m,m) \in R_1$\\ Da $R_1 \subseteq R_2 \rightarrow \forall m \in M : (m,m) \in R_2$\\ Damit ist $R_2$ auch reflexiv.
			\item[ii)]
				$R_1 $ symmetrisch $\Rightarrow R_2 $ symmetrisch.\\
				$\forall m,n \in M : (m,n) \in R_1 \Rightarrow (n,m) \in R_1$\\
				Gegenbeispiel: $M := \{a,b,c\}$\\
				$R_1 := \{(a,b),(b,a)\}$\\
				$R_2 := \{(a,b),(b,a),(c,a)\}$\\
				$\forall m,n \in R_1 : (m,n) \in R_1 \Rightarrow (n,m) \in R_1$\\
				$\forall m,n \in R_2 $ gilt die Symmetrie nicht, da $(a,c) \in R_2$ aber $(c,a) \notin R_2 $\\
				
				Also folgt aus 	$R_1 $ symmetrisch nicht, dass auch $R_2$ symmetrisch.
				\item[iii)]
		\end{itemize}
\end{itemize}






























\end{document}