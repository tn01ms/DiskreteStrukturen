
\documentclass[paper = a4, ngerman]{scrartcl}

\usepackage[super,sort&compress]{natbib} 

\usepackage{varioref}

\usepackage{hyperref}

\usepackage[T1]{fontenc}

\usepackage{makeidx}
\makeindex

\usepackage{listings}
\usepackage{color}

\definecolor{dkgreen}{rgb}{0,0.6,0}
\definecolor{gray}{rgb}{0.5,0.5,0.5}
\definecolor{mauve}{rgb}{0.58,0,0.82}
\definecolor{light-gray}{gray}{0.95}
\lstset{frame=single,
	language=Java,
	aboveskip=3mm,
	belowskip=3mm,
	showstringspaces=false,
	columns=flexible,
	basicstyle={\small\ttfamily},
	numbers=none,
	backgroundcolor=\color{light-gray},
	numberstyle=\tiny\color{gray},
	keywordstyle=\color{blue},
	commentstyle=\color{dkgreen},
	stringstyle=\color{mauve},
	breaklines=true,
	breakatwhitespace=true,
	tabsize=3
}
\usepackage[miktex]{gnuplottex}

\usepackage{amsmath,amssymb}

\usepackage[noabbrev]{cleveref}

\usepackage{amsfonts}

\setlength{\parindent}{0pt} %Nicht Einrücken

\usepackage{color}


\usepackage[ngerman]{babel}

\usepackage[utf 8]{inputenc}

\usepackage{float}

\usepackage{graphicx}

\usepackage{alltt}


\subject{Diskrete Strukturen}
\title{Blatt 4}
\author{Eike Janning, Tobias Nagel, Jan Paul Wessendorf}
\date{Abgabe: 31.05.2019}

\renewcommand{\arraystretch}{1.25}



\begin{document}
	\maketitle
	\hrulefill
	
	\section*{Aufgabe 16}
	
	\textbf{Behauptung:}\\
	Zwei Mengen sind identisch $\Leftrightarrow$ Ihre Potenzmengen sind identisch\\
	
	$\Rightarrow A = B \Leftrightarrow \mathcal{P}(A) = \mathcal{P}(B) $\\
	
	\textbf{Beweis:}\\
	$ A = B \Rightarrow \mathcal{P}(A) = \mathcal{P}(B) $\\
	Trivial. Geht aus Definition der Potenzmenge hervor: $\mathcal{P}(A) := \{ X | X \subseteq A\}$\\
	
	$\mathcal{P}(A) = \mathcal{P}(B) \Rightarrow A = B$\\
	
	Durch die Definition der Potenzmenge folgt, dass $\mathcal{P}(A)$ die Menge aller Teilmengen von $A$ ist. Weiterhin gilt, dass $A$ auch eine Teilmenge von sich selbst ist.\\
	D.h. $A \subseteq \mathcal{P}(A)$\\
	Da $\mathcal{P}(A) = \mathcal{P}(B)$ gilt, folgt daraus: $A \subseteq B$  (Da $\mathcal{P}(B)$ die Menge aller Teilmengen von B ist)\\
	
	Andersherum folgt genauso:\\
	$B \subseteq \mathcal{P}(B)$\\
	Da $\mathcal{P}(A) = \mathcal{P}(B)$ gilt, folgt daraus: $B \subseteq A$\\
	
	Aus $A \subseteq B$ und $B \subseteq A$ folgt $A = B$\\
	
	Also gilt 	$\Rightarrow A = B \Leftrightarrow \mathcal{P}(A) = \mathcal{P}(B) $\\
	
	\hfil$\Box$
	\section*{Aufgabe 17}
	Es sei $(S,\preccurlyeq)$ eine Partielle Ordnung, in der je zwei Elemente $x,y\in S$ ein Infimum und ein Supremum besitzen und in der für alle $x,y,z \in S$ gilt:
	\begin{align}
		x \wedge (y \vee z) = (x \wedge y) \vee (x \wedge y)
	\end{align}
	Folgern Sie daraus (ohne Folie 1.39 zu verwende), dass auch $x \vee (y \wedge z) = (x \vee y) \wedge (x \vee y)$ für alle $x,y,z \in S$ gilt.
	
	
	
	\section*{Aufgabe 18}
	
	Sei $G = (S,\diamond)$ eine Gruppe mit neutralem Element $e$.\\
	$a^{-1}$ sei das Inverse Element zum Element $a \in S$.\\
	
	\subsection*{a)}
	\textbf{Behauptung:} $\forall a,b \in S$ gilt $ (a \diamond b)^{-1} = b^{-1} \diamond a^{-1}$\\
	
	\textbf{Beweis:}\\
	$ (a \diamond b)^{-1} = b^{-1} \diamond a^{-1}$\\
	$\Leftrightarrow (a \diamond b)^{-1} \diamond (a \diamond b)= b^{-1} \diamond a^{-1} \diamond(a \diamond b)$\\
	$\Leftrightarrow (a \diamond b)^{-1} \diamond (a \diamond b)= (b^{-1} \diamond (a^{-1} \diamond a)) \diamond b$ \hspace{10mm} (Assoziativität von G)\\
	$\Leftrightarrow (a \diamond b)^{-1} \diamond (a \diamond b)= (b^{-1} \diamond e) \diamond b$\\
	$\Leftrightarrow (a \diamond b)^{-1} \diamond (a \diamond b)= b^{-1} \diamond b$\\
	$\Leftrightarrow (a \diamond b)^{-1} \diamond (a \diamond b)= e$\\
	$\Leftrightarrow e= e$\\
	
	Also gilt die Behauptung $\forall a,b \in S$ gilt $ (a \diamond b)^{-1} = b^{-1} \diamond a^{-1}$
	\hfill$\Box$\\
	
	\subsection*{b)}
	\textbf{Behauptung:} $\forall a,b \in S$ gilt $ (a \diamond b)^{-1} = a^{-1} \diamond b^{-1}$\\
	
	\textbf{Wiederlegung:}\\
	$ (a \diamond b)^{-1} = a^{-1} \diamond b^{-1}$  (Teilaufgabe a))\\
	$\Leftrightarrow b^{-1} \diamond a^{-1} = a^{-1} \diamond b^{-1}$\\
	
	Dies stimmt nicht, da die Eigenschaft $\forall x, y \in S: x \diamond y = y \diamond x$ nur in abelschen Gruppen gilt.
	\hfill$\Box$\\
		
	\subsection*{c)}
	\textbf{Behauptung:} Falls $\forall a \in S: a \diamond a = e $, dann ist G abelsch.\\
	
	\textbf{Beweis:}\\
	In abelschen Gruppen gilt die Kommutativität, d.h.: $\forall a, b \in S: a \diamond b = b \diamond a$\\
	
	Zu zeigen: $\forall a, b \in S: a \diamond b = b \diamond a$\\
	
	$a \diamond b = b \diamond a$\\
	$\Leftrightarrow a \diamond b \diamond a \diamond b = b \diamond a\diamond a \diamond b$ \hspace{10mm}(Assoziativität von G)\\
	$\Leftrightarrow (a \diamond b) \diamond (a \diamond b) = (b \diamond (a\diamond a)) \diamond b$  \\
	$\Leftrightarrow e = (b \diamond e) \diamond b$  \\
	$\Leftrightarrow e = b \diamond b$  \\
	$\Leftrightarrow e = e$\\
	
	Also gilt die $\forall a, b \in S: a \diamond b = b \diamond a$. Damit gilt auch die Behauptung und G ist abelsch.
		\hfill$\Box$\\
	
\section*{Aufgabe 19}
Sei S eine Menge und $\diamond: S \times S \to S$. Für die Algebra $(S, \diamond)$ sind folgende Bedingungen erfüllt:\\

1. Assoziativität: $\forall x, y,z \in S: (x \diamond y) \diamond z = x \diamond (y \diamond z)$\\
2. Linksneutrales Element: es gibt $e \in S$ mit $\forall x \in S: e \diamond x = x$\\
3. Linksinverses Element: $\forall x \in S: \exists y \in S: y\diamond x = e$\\

Zeigen Sie wie folgt, dass $(S, \diamond)$ eine Gruppe ist:

\subsection*{a)}
Seien $x,y,z \in S$. Angenommen, es gilt $y \diamond x = e $ und  $z \diamond y = e$\\
Zu zeigen: $x \diamond y = (z \diamond y) \diamond (x \diamond y)$\\

Beweis:\\
$x \diamond y = (z \diamond y) \diamond (x \diamond y)$\\
$\Leftrightarrow x \diamond y = e \diamond (x \diamond y)$ \hspace{10mm} (Bedingung 2 ) \\
$\Leftrightarrow x \diamond y = x \diamond y$	\hfill$\Box$\\

\subsection*{b)}
Zu zeigen: $x \diamond y = e$ (d.h. jedes linksinverse Element der Gruppe ist auch rechtsinvers).\\

Beweis:\\
$x \diamond y = e$ \hspace{10mm} (Aussage a)\\
$\Leftrightarrow (z \diamond y) \diamond (x \diamond y) = e$ \hspace{10mm} (Assoziativität)\\
$\Leftrightarrow z \diamond ((y \diamond x) \diamond y) = e$ \hspace{10mm} (Annahme in a)\\
$\Leftrightarrow z \diamond (e \diamond y) = e$ \hspace{10mm} (Bedingung 3)\\
$\Leftrightarrow z \diamond y = e$ \hspace{10mm} (Annahme in a)\\
$\Leftrightarrow e = e$\hfill$\Box$\\

\subsection*{c)}
Zu Zeigen: $x \diamond e = x$ (d.h. jedes linksneutrale Element der Gruppe ist auch rechtsneutral)\\

Beweis:\\
$x \diamond e = x$\\ \hspace{10mm} (Annahme in a)\\
$\Leftrightarrow x \diamond (y \diamond x ) = x$ \hspace{10mm} (Assoziativität)\\ 
$\Leftrightarrow (x \diamond y) \diamond x  = x$ \hspace{10mm} (Aussage b)\\ 
$\Leftrightarrow e \diamond x  = x$ \hspace{10mm} (Bedingung 2))\\ 
$\Leftrightarrow x  = x$ \hfill$\Box$\\

Da nun jedes linksinverse Element auch rechtsinvers und jedes linksneutrale Element auch rechtsneutral ist und die Assoziativität in Bedingung 1 gegeben ist, handelt es sich bei $(S, \diamond)$ um eine Gruppe.\hfill$\Box$\\


	
	
\section*{Aufgabe 20}
Zeigen Sie, dass jede Teilmenge B einer abzählbaren, d.h. endlich oder abzählbar unendlichen, Menge A ebenfalls abzählbar ist.
\textbf{Beweis:}
\begin{itemize}
	\item Fall 1 : A endlich
\end{itemize}
Da A endlich und $B \subseteq A$  folgt das B maximal gleich A sein kann und damit auch endlich sein muss. Endlich viele Elemente kann man zählen, also ist B abzählbar.
\begin{itemize}
	\item Fall 2 : A abzählbar unendlich
\end{itemize}
Da A abzählbar unendlich und $B \subseteq A$  folgt das B maximal gleich A sein kann.\\
Falls B nun endlich ist der Beweis der Abzählbarkeit trivial.(Genauso wie wenn A endlich)\\

Falls B nun nicht endlich:\\
Da A abzählbar unendlich und $B \subseteq A$  folgt, dass B unendlich und maximal gleich A sein kann.\\
 Sei $b_i \in B$ def. als das kleinste Element von B, dann bilden wir das erste Element von den natürlichen Zahlen $n_i \in \mathbb{N}_{0}$ auf dieses $b_i$ ab. Für das nächste Element aus B, $b_{i+1}$ nehmen wir den Nachfolger aus $\mathbb{N}_{0}\; n_{i+1}$.\\\\
 Diese Abbildung $f: \mathbb{N} \rightarrow B$ ist bijektiv weil:\\
 Das Abzählen eine Menge ist Injektiv und Surjektiv (und damit auch Bijektiv), weil sich der Wert der Definitionsmenge immer um eins erhöht und gleichzeitig auch der Wert der Wertemenge was bedeutet, dass:\\ 1. nicht 2 Elemente aus der Definitionsmenge auf dasselbe aus der Wertemenge zeigen können\\
 2. das jedes Element aus der Wertemenge getroffen wird. 
 
	

	
	
\end{document}