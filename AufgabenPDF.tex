
\documentclass[paper = a4, ngerman]{scrartcl}

\usepackage[super,sort&compress]{natbib} 

\usepackage{varioref}

\usepackage{hyperref}

\usepackage[T1]{fontenc}

\usepackage{makeidx}
\makeindex

\usepackage{listings}
\usepackage{color}

\definecolor{dkgreen}{rgb}{0,0.6,0}
\definecolor{gray}{rgb}{0.5,0.5,0.5}
\definecolor{mauve}{rgb}{0.58,0,0.82}
\definecolor{light-gray}{gray}{0.95}
\lstset{frame=single,
	language=Java,
	aboveskip=3mm,
	belowskip=3mm,
	showstringspaces=false,
	columns=flexible,
	basicstyle={\small\ttfamily},
	numbers=none,
	backgroundcolor=\color{light-gray},
	numberstyle=\tiny\color{gray},
	keywordstyle=\color{blue},
	commentstyle=\color{dkgreen},
	stringstyle=\color{mauve},
	breaklines=true,
	breakatwhitespace=true,
	tabsize=3
}
%\usepackage[miktex]{gnuplottex}

\usepackage{amsmath,amssymb}

\usepackage[noabbrev]{cleveref}

\usepackage{amsfonts}

\setlength{\parindent}{0pt} %Nicht Einrücken

\usepackage{color}


\usepackage[ngerman]{babel}

\usepackage[utf 8]{inputenc}

\usepackage{float}

\usepackage{graphicx}

\usepackage{alltt}


\subject{Diskrete Strukturen}
\title{Blatt 1}
\author{Eike Janning, Tobias Nagel, Jan Paul Wessendorf}
\date{Abgabe: 19.04.2019}

\renewcommand{\arraystretch}{1.25}



\begin{document}
\maketitle
\hrulefill

\section*{Aufgabe 1}

Geben Sie eine induktive Definition der Tiefe eines Baums an.\\
$T_W$ := \text{Die Tiefe des Baumes mit der Wurzel} $_W$

	\begin{align*}
		T_W =
		\begin{cases}
			\text{Keine verbundenen Unterknoten} &: 0\\
			\text{Sonst} &: max(T_{W_1}, T_{W_2}, ..., T_{W_k}) + 1
		\end{cases}
	\end{align*}\\

\section*{Aufgabe 2}
	 Geben Sie eine induktive Definition für arithmetische Ausdrücke mit Multiplikation und Addition (einschließlich Klammersetzung) für die natürlichen Zahlen an.\\
	 	\begin{align*}
		 	a, b_n \in \mathbb{N}, a = 0, b_1 = 1, b_n = n*(a+b_1)
	 	\end{align*}\\
	 	
\pagebreak

\section*{Aufgabe 3}
Vorrechnen! Definieren Sie induktiv die Menge der (Nummern der) Stufen, die der Briefträger auf seinem Weg nach oben auf dieser Treppe erreichen kann.\\
S sei die Menge der erreichbaren Stufen
\begin{itemize}
	\item $1 \in S$
	\item $\forall s \in S, s+2 \le 60 : s + 2 \in S$
	\item $\forall s \in S, s+3 \le 60 : s + 3 \in S$
\end{itemize}

\section*{Aufgabe 4} Vorrechnen!
Sei $M_i$ eine Menge, wobei  so geben Sie jeweils die Elemente der Mengen $M_3$, $M_{17}$ und $M_{42}$ an.\\\\
	1. $ i \in M$\\
	2. $m \in M \land m\: \textit{\textbf{ungerade}} \Rightarrow (3m+1) \in M$\\
	3. $m \in M \land m\: \textit{\textbf{gerade}} \Rightarrow (\frac{m}{2}) \in M$
		\begin{align*}
			M_3 &=: \left\{1,2,3,4,5,8,10,16\right\}\\
			M_{17} &=: \left\{1,2,4,5,8,10,13,16,17,20,26,40,52\right\}\\
			M_{42} &=: \left\{1,2,4,8,16,21,32,42,64\right\}
		\end{align*}
		
		
\section*{Aufgabe 5} Vorrechnen!
	 Geben Sie zu jeder der nachfolgenden Aussagen an, ob sie korrekt ist.
		\begin{align*}
			\varnothing &= \varnothing \quad \textit{Wahr}\\
			\varnothing &\subseteq \varnothing \quad \textit{Wahr}\\
			\varnothing &\subsetneq \varnothing \quad \textit{Falsch}\\
			\varnothing &\in \varnothing \quad \textit{Falsch}\\
			\varnothing &\subseteq \left\{\varnothing\right\} \quad \textit{Wahr}\\
			\varnothing &\subsetneq \left\{\varnothing\right\} \quad \textit{Wahr}\\
		\end{align*}
		
\pagebreak

\section*{Aufgabe 6} Vorrechnen!

\subsection*{a)}
$(A \cap B^C)^C$ (Regeln von de Morgan)\\
$= A^C \cup (B^C)^C$ (Involution)\\
$= A^C \cup B$\\

\subsection*{b)}
$(((A\cap B) \cup C)^C \cap B^C)^C$ (Regeln von de Morgan)\\
$= (((A\cap B)\cup C)^C)^C \cup (B^C)^C$ (Involution)\\
$= ((A\cap B)\cup C) \cup B$ (Assoziativgesetz)\\
$= (A\cap B) \cup (C \cup B)$ (Kommutativgesetz)\\
$= (B\cap A) \cup (B \cup C) $ (Assoziativgesetz)\\
$= ((B\cap A)\cup B) \cup C$  (Kommutativgesetz)\\
$= (B\cup (B\cap A))\cup C$ (Absorptionsgesetz)\\
$= B \cup C$



\end{document}
