
\documentclass[paper = a4, ngerman]{scrartcl}

\usepackage[super,sort&compress]{natbib} 

\usepackage{varioref}

\usepackage{hyperref}

\usepackage[T1]{fontenc}

\usepackage{makeidx}
\makeindex

\usepackage{listings}
\usepackage{color}

\definecolor{dkgreen}{rgb}{0,0.6,0}
\definecolor{gray}{rgb}{0.5,0.5,0.5}
\definecolor{mauve}{rgb}{0.58,0,0.82}
\definecolor{light-gray}{gray}{0.95}
\lstset{frame=single,
	language=Java,
	aboveskip=3mm,
	belowskip=3mm,
	showstringspaces=false,
	columns=flexible,
	basicstyle={\small\ttfamily},
	numbers=none,
	backgroundcolor=\color{light-gray},
	numberstyle=\tiny\color{gray},
	keywordstyle=\color{blue},
	commentstyle=\color{dkgreen},
	stringstyle=\color{mauve},
	breaklines=true,
	breakatwhitespace=true,
	tabsize=3
}
\usepackage[miktex]{gnuplottex}

\usepackage{amsmath,amssymb}

\usepackage[noabbrev]{cleveref}

\usepackage{amsfonts}

\setlength{\parindent}{0pt} %Nicht Einrücken

\usepackage{color}


\usepackage[ngerman]{babel}

\usepackage[utf 8]{inputenc}

\usepackage{float}

\usepackage{graphicx}

\usepackage{alltt}


\subject{Diskrete Strukturen}
\title{Blatt [NR]}
\author{Eike Janning, Tobias Nagel, Jan Paul Wessendorf}
\date{Abgabe: [TAG].[MONAT].2019}

\renewcommand{\arraystretch}{1.25}



\begin{document}
\maketitle
\hrulefill
\section*{Aufgabe 12:}
Sei M eine Menge mit zehn paarweise verschiedenen zweistelligen Zahlen.\\

Zu Zeigen: Es gibt zwei nicht-leere disjunkte Teilmengen $M_1$ und $M_2$ von M, sodass die Summen der Elemente in $M_1$ und $M_2$ gleich ist.\\

Beweis: (Schubladenprinzip)\\

Wenn man M in zwei nicht-leere Teilmengen aufteilt, so kann jede dieser Teilmengen maximal 9 Elemente enthalten.\\

Die kleinste Zahl, die man mit so einer Teilmenge darstellen kann ist die 10.\\

Die größte Zahl, die man darstellen kann, bildet sich aus einer Menge von 9 Elementen:\\
$91+92+93+94+95+96+97+98+99 = 855$\\

Es gibt also 855 - 10 = 845 verschiedene Zahlen, die dargestellt werden können.\\

Nun betrachtet man die Anzahl der Kombinationen, die es gibt um eine Teilmenge von M zu bilden:\\

Um eine Menge mit einem Element zu bilden, gibt es 10 Möglichkeiten.\\
Um eine Menge mit  zwei Elementen zu bilden, gibt es $\binom{10}{2} = 45$ Möglichkeiten.\\

Um eine Menge mit 1,2,3,4,5,6,7,8 oder 9 Elementen zu bilden, gibt es also folgendermaßen viele Möglichkeiten:\\

$\binom{10}{1} + \binom{10}{2} +\binom{10}{3} +\binom{10}{4} + \binom{10}{5} +\binom{10}{6} +\binom{10}{7} + \binom{10}{8} +\binom{10}{9} = 1022$\\

Man sieht also, dass es mehr Möglichkeiten gibt eine Teilmenge von M (mit 1-9 Elementen) zu bilden, als es mögliche Summen der einzelnen Elemente gibt. Demnach muss es Summen geben, die mehrfach vorkommen.\\
\hfil$\Box$
\section*{Aufgabe 13:}
Vorrechnen!\\

Menge M, $|M| = n, R, R_1, R_2 $ Relationen auf M\\

\begin{itemize}
	\item[a)]
		$\rightarrow |R| \ge n \Rightarrow $ R reflexiv\\
		$M := \{a,b\} \Rightarrow |M| = n = 2$\\
		$M \times M = \{(a,a),(a,b),(b,a),(b,b)\}$\\
		Da $|R| \ge n$ folgt, da $n = 2$, dass $|R| \ge 2$\\
		Da $R \subseteq M \times M$ gilt z.B.: $R := \{(a,b),(b,a)\}$\\
		Damit ist die Reflexivität nicht erfüllt, da $\forall m \in M : (m,m) \in R$.\\
	\item[b)]
		$R_1 \subseteq R_2$
		\begin{itemize}
			\item[i)]
				$R_1$ reflexiv $\Rightarrow R_2 $ reflexiv\\
				$\rightarrow \forall m \in M : (m,m) \in R_1$\\ Da $R_1 \subseteq R_2 \rightarrow \forall m \in M : (m,m) \in R_2$\\ Damit ist $R_2$ auch reflexiv.
			\item[ii)]
				$R_1 $ symmetrisch $\Rightarrow R_2 $ symmetrisch.\\
				$\forall m,n \in M : (m,n) \in R_1 \Rightarrow (n,m) \in R_1$\\
				Gegenbeispiel: $M := \{a,b,c\}$\\
				$R_1 := \{(a,b),(b,a)\}$\\
				$R_2 := \{(a,b),(b,a),(c,a)\}$\\
				$\forall m,n \in R_1 : (m,n) \in R_1 \Rightarrow (n,m) \in R_1$\\
				$\forall m,n \in R_2 $ gilt die Symmetrie nicht, da $(a,c) \in R_2$ aber $(c,a) \notin R_2 $\\
				
				Also folgt aus 	$R_1 $ symmetrisch nicht, dass auch $R_2$ symmetrisch.
\pagebreak
			\item[iii)]
				$R_1$ antisymmetrisch $\Rightarrow R_2$ antisymmetrisch.\\
				$\forall m,n \in R_1 : (m,n) \in R_1 \wedge (n,m) \in R_1 \Rightarrow m = n$\\
				Gegenbeispiel:\\
				$M := \{a,b,c\}$\\
				$R_1 := \{(a,b),(b,a)\}$\\
				$R_2 := \{(a,b),(b,a),(a,c),(c,a)\}$\\\\
				$\forall m,n \in R_2 $ gilt die Antisymmetrie nicht zwingend, da wir keine Aussage über $ c = a $ treffen können.\\
				
				Also folgt aus 	$R_1 $ antisymmentrisch nicht, dass auch $R_2$ antisymmentrisch.
			\item[iv)]
				$\forall m,n,o \in R_1 : (m,n) \in R_1 \wedge (n,o) \in R_1 \Rightarrow (m,o) \in R_1$\\
				Gegenbeispiel:\\
				$M := \{a,b,c,d,e,f\}$\\
				$R_1 := \{(a,b),(b,c),(a,c)\}$\\
				$R_2 := \{(a,b),(b,c),(a,c),(d,e),(e,f))\}$\\
				$\Rightarrow \forall m,n,o \in R_1 : (m,n) \in R_1 \wedge (n,o) \in R_1 \Rightarrow (m,o) \in R_1$\\
				$R_2$ ist nicht transitiv, da $(d,e), (e,f) \in R_2 $ aber $(d,f)) \notin R_2$\\
				
				Also folgt aus 	$R_1 $ transitiv nicht, dass auch $R_2$ transitiv.
		\end{itemize}
	\item[c)]
		$R$ ist eine Äquivalenzrelation $\Rightarrow R$ ist reflexiv, symmetrisch, transitiv\\
		Zu zeigen: $n \le |R| \le n^2$\\
		$M \times M $ hat $n^2$ Elemente.\\ Da $R \le M \times M$, kann $|R|$ maximal $n^2$ Elemente haben.\\\\
		Reflexivität von $R \Rightarrow \forall m \in M : (m,m) \in R$\\
		Da M n Elemente hat muss damit R mindestens auch n Elemente haben, also folgt:\\
		$n \le |R| \le n^2$
\end{itemize}




\end{document}