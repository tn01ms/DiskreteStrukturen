
\documentclass[paper = a4, ngerman]{scrartcl}

\usepackage[super,sort&compress]{natbib} 

\usepackage{varioref}

\usepackage{hyperref}

\usepackage[T1]{fontenc}

\usepackage{makeidx}
\makeindex

\usepackage{listings}
\usepackage{color}

\definecolor{dkgreen}{rgb}{0,0.6,0}
\definecolor{gray}{rgb}{0.5,0.5,0.5}
\definecolor{mauve}{rgb}{0.58,0,0.82}
\definecolor{light-gray}{gray}{0.95}
\lstset{frame=single,
	language=Java,
	aboveskip=3mm,
	belowskip=3mm,
	showstringspaces=false,
	columns=flexible,
	basicstyle={\small\ttfamily},
	numbers=none,
	backgroundcolor=\color{light-gray},
	numberstyle=\tiny\color{gray},
	keywordstyle=\color{blue},
	commentstyle=\color{dkgreen},
	stringstyle=\color{mauve},
	breaklines=true,
	breakatwhitespace=true,
	tabsize=3
}
\usepackage[miktex]{gnuplottex}

\usepackage{amsmath,amssymb}

\usepackage[noabbrev]{cleveref}

\usepackage{amsfonts}

\setlength{\parindent}{0pt} %Nicht Einrücken

\usepackage{color}


\usepackage[ngerman]{babel}

\usepackage[utf 8]{inputenc}

\usepackage{float}

\usepackage{graphicx}

\usepackage{alltt}


\subject{Diskrete Strukturen}
\title{Blatt 2}
\author{Eike Janning, Tobias Nagel, Jan Paul Wessendorf}
\date{Abgabe: 03.05.2019}

\renewcommand{\arraystretch}{1.25}



\begin{document}
	\maketitle
	\hrulefill
	
	\section*{Aufgabe 7}
	\textit{Vorrechnen!}
		\begin{itemize}
			\item[a)] Zu zeigen: $\forall x,y \in \mathbb{N}_0 \backslash \{ 0 \}$, x,y ungerade $: x*y$ ungerade\\
			Laut Definition von ungeraden Zahlen:\\
			$x = 2*m+1$, $m\in \mathbb{N}_0$\\
			$y = 2*n+1$, $n\in \mathbb{N}_0$
			\begin{center}
				$\Rightarrow x*y = (2*m+1)*(2*n+1)$\\
				$\Rightarrow x*y = 4mn + 2m + 2n + 1$
			\end{center}
			Da $4mn$, $2m$ und $2n$ gerade (da 2|2 und 2|4) sowie $1$ ungerade, ist $x*y$ ungerade.
			
			\item[b)] Zu zeigen: $\forall x \in \mathbb{R} : x^2 - 4x + 5 > 0$\\
			Mit binomischer Formel: $x^2 - 4x + 5 > 0 \Leftrightarrow (x - 2)^2 + 1 > 0$\\\\
			Da jede quadrierte Zahl $\ge 0$ ist und $1 > 0$ ist der Beweis trivial.
			
			\item[c)]
			Beweis überdeckende Fallunterscheidung:\\
			Für jedes $n \in \mathbb{N}_0 / \{ 0 \}$ gilt:\\
			$n+1$ ungerade $\Rightarrow n*(n+1)*(n+2)$ teilbar durch 24.\\
			Es gilt: $24*x$ teilbar durch 24, da dies ein Vielfaches von 24 ist.\\
			Fälle $n<24$:
			\begin{align*}
				n=2 &\Rightarrow 2*3*4 = 24\\
				n=4 &\Rightarrow 4*5*6 = 120 = 5*24\\
				n=6 &\Rightarrow 6*7*8 = 48*7 = 2*7*24\\
				n=8 &\Rightarrow 8*9*10 = 720 = 30*24\\
				n=10 &\Rightarrow 10*11*12 = 12*110 = 55*24\\
				n=12 &\Rightarrow 12*13*14 = 12*182 = 91*24\\
				n=14 &\Rightarrow 14*15*16 = 12*18214*3*5*2*8 = 14*10*24\\
				n=16 &\Rightarrow 16*17*18 = 2*8*17*3*6 = 24*12*17\\
				n=18 &\Rightarrow 18*19*20 = 3*6*19*4*5 = 24*15*19\\
				n=20 &\Rightarrow 20*21*22 = 4*5*3*7*2*11 = 24*11*3*7\\
				n=22 &\Rightarrow 22*23*24 = 22*23*24\\
				n=24 &\Rightarrow 24*25*26 = 24*25*26\\
			\end{align*}
			
			Fälle $n>24$:\\
				Für alle möglichen $n>24$ betrachte $n\ mod\ 24$.\\
				Bsp: $n=26$\\
				$n=26 \Rightarrow 26*27*28 = 2*13*9*3*7*4$
				Diese Multiplikation enthält die Faktoren 2,3,4 wie bei $n=2$
		\end{itemize}
	
	
	\section*{Aufgabe 8}
		\begin{itemize}
			\item[a)] Fallunterscheidung:\\
			Fall 1 (von x und y ist eine Zahl gerade):
			\begin{center}
				$(2m + 1) * 2n = 4mn + 2n = 2*(2mn + n) = z $\\$\Rightarrow$ z gerade da $2|2*(2mn + n)$
			\end{center}
			Fall " (von x und y beide gerade):
			\begin{center}
				$2m * 2n = 2*(2mn) = z \Rightarrow$ z gerade da $2|2*(2mn)$
			\end{center}
		\end{itemize}
	
	\pagebreak
	\section*{Aufgabe 9}
	\textit{Vorrechnen!}
		\begin{itemize}
			\item[a)] Im Induktionsschritt wird angenommen dass $a=0$ und $b=0$ sind um den Induktionsanfang zu verwenden.
			\item[b)] Zu zeigen: $\displaystyle\sum_{i=0}^{n} i*2^i = (n-1)*2^{n+1}+2$\\
			IA (n=0):
			\begin{align*}
			0*2^0 = 0*1 = 0 = (-2) + 2 = (0-1)*2^{0+1}+2 \hspace{3mm}\checkmark
			\end{align*}
			IV: Für ein beliebiges aber festes n gelte $\displaystyle\sum_{i=0}^{n} i*2^i = (n-1)*2^{n+1}+2$\\
			IS (n+1):
			\begin{align*}
			\displaystyle\sum_{i=1}^{n+1} (n+1)*2^{n+1} = \sum_{i=0}^{n} i*2^i + (n+1)*2^{n+1} &\overset{\text{\tiny{IV}}}{=}\\
			(n-1)*2^{n+1}+2+(n+1)*2^{n+1} &=\\
			(n-1)*2*2^{n}+2+(n+1)*2*2^n &=\\
			(2n-2)*2^{n}+2+(2n+2)*2^n &=\\
			(2n-2)*2^{n}+(2n+2)*2^n + 2 &=\\
			2^{n+1}*n-2^{n+1}+2^{n+1}*n+2^{n+1}+2 &=\\
			2^{n+1}*n+2^{n+1}*n+2 &=\\
			2*(2^{n+1}*n)+2 &=\\
			2^{n+2}*n+2 &\\
			\text{Entspricht IV mit } n=n+1& \hspace{6mm} \Box
			\end{align*}
		\end{itemize}
	

\pagebreak
	\section*{Aufgabe 11}
		\textit{Vorrechnen!}
		Voraussetzung: Es gibt in jedem Haus eine Person die jeweils einer Person aus den anderen beiden Häusern vertraut.\\
		Es sind zu jeder Zeit $n$ Personen in jedem Haus und jede Person vertraut insgesamt $n+1$ anderen Personen aus beiden Häusern. Da $n+1 > n$ gilt, muss es mindestens eine Person geben die in einem anderen Haus ist der man vertraut, auch wenn alle anderen vertrauten Personen in dem verbleibenden Haus sind.\\
		Da der Zusammenschluss aus 3 Häusern entstehen soll, gibt es somit mindestens 3 Personen für die die Voraussetzung gilt.
\end{document}