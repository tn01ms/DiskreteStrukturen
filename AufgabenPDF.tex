
\documentclass[paper = a4, ngerman]{scrartcl}

\usepackage[super,sort&compress]{natbib} 

\usepackage{varioref}

\usepackage{hyperref}

\usepackage[T1]{fontenc}

\usepackage{makeidx}
\makeindex

\usepackage{listings}
\usepackage{color}

\definecolor{dkgreen}{rgb}{0,0.6,0}
\definecolor{gray}{rgb}{0.5,0.5,0.5}
\definecolor{mauve}{rgb}{0.58,0,0.82}
\definecolor{light-gray}{gray}{0.95}
\lstset{frame=single,
	language=Java,
	aboveskip=3mm,
	belowskip=3mm,
	showstringspaces=false,
	columns=flexible,
	basicstyle={\small\ttfamily},
	numbers=none,
	backgroundcolor=\color{light-gray},
	numberstyle=\tiny\color{gray},
	keywordstyle=\color{blue},
	commentstyle=\color{dkgreen},
	stringstyle=\color{mauve},
	breaklines=true,
	breakatwhitespace=true,
	tabsize=3
}
\usepackage[miktex]{gnuplottex}

\usepackage{amsmath,amssymb}

\usepackage[noabbrev]{cleveref}

\usepackage{amsfonts}

\setlength{\parindent}{0pt} %Nicht Einrücken

\usepackage{color}


\usepackage[ngerman]{babel}

\usepackage[utf 8]{inputenc}

\usepackage{float}

\usepackage{graphicx}

\usepackage{alltt}


\subject{Diskrete Strukturen}
\title{Blatt 5}
\author{Eike Janning, Tobias Nagel, Jan Paul Wessendorf}
\date{Abgabe: 21.06.2019}

\renewcommand{\arraystretch}{1.25}



\begin{document}
	\maketitle
	\hrulefill
	
	\section*{Aufgabe 25}
	
	Bestimmen Sie die Anzahl der durch 2, 3 oder 5 teilbaren natürlichen Zahlen (einschließlich Null) kleiner gleich 100.\\
	
	Es gibt offensichtlich $\frac{100}{2} = 50$ verschiedene Zahlen von 1 bis 100, die durch 2 teilbar sind.
	Dasselbe kann man auch für $\frac{100}{3}$ betrachten. Da hier allerdings eine Kommazahl herauskommt, kann man hier die Floor-Funktion zu Hilfe nehmen: \\
	$\Rightarrow$ es gibt $floor(\frac{100}{3}) = 33$ verschiedene Zahlen von 1 bis 100, die durch 3 teilbar sind.\\
	Dasselbe gilt auch für die 5.\\
	
	$\Rightarrow floor(\frac{100}{2}) + floor(\frac{100}{3}) +  floor(\frac{100}{5}) = 103$\\
	
	Dies sind offensichtlich zu viele Zahlen, da manche Zahlen doppelt auftauchen. Z.B ist die 6 sowohl in den durch 2, als auch in den durch 3 teilbaren Zahlen enthalten. Diese Zahlen müssen wir entsprechend abziehen (6 = 3*2, 10 = 5*2, 15 = 5*3):\\
	
	$\Rightarrow floor(\frac{100}{2}) + floor(\frac{100}{3}) +  floor(\frac{100}{5}) - floor(\frac{100}{6}) - floor(\frac{100}{10}) - floor(\frac{100}{15})= 71$\\
	
	Hierzu müssen wir wiederum die Zahlen addieren, die wir nun doppelt abgezogen haben (alle Zahlen die durch 2*3*5 = 30 teilbar sind). Außerdem müssen wir die Null addieren (+1):\\
	
		$floor(\frac{100}{2}) + floor(\frac{100}{3}) +  floor(\frac{100}{5})$\\$ - floor(\frac{100}{6}) - floor(\frac{100}{10}) - floor(\frac{100}{15}) + floor(\frac{100}{30}) + 1 = 75$\\
	
	$\Rightarrow$ Also gibt es 75 Zahlen von 0 bis 100, die durch 2,3 oder 5 teilbar sind.
\end{document}