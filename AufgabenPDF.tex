
\documentclass[paper = a4, ngerman]{scrartcl}

\usepackage[super,sort&compress]{natbib} 

\usepackage{varioref}

\usepackage{hyperref}

\usepackage[T1]{fontenc}

\usepackage{makeidx}
\makeindex

\usepackage{listings}
\usepackage{color}

\definecolor{dkgreen}{rgb}{0,0.6,0}
\definecolor{gray}{rgb}{0.5,0.5,0.5}
\definecolor{mauve}{rgb}{0.58,0,0.82}
\definecolor{light-gray}{gray}{0.95}
\lstset{frame=single,
	language=Java,
	aboveskip=3mm,
	belowskip=3mm,
	showstringspaces=false,
	columns=flexible,
	basicstyle={\small\ttfamily},
	numbers=none,
	backgroundcolor=\color{light-gray},
	numberstyle=\tiny\color{gray},
	keywordstyle=\color{blue},
	commentstyle=\color{dkgreen},
	stringstyle=\color{mauve},
	breaklines=true,
	breakatwhitespace=true,
	tabsize=3
}
\usepackage[miktex]{gnuplottex}

\usepackage{amsmath,amssymb}

\usepackage[noabbrev]{cleveref}

\usepackage{amsfonts}

\setlength{\parindent}{0pt} %Nicht Einrücken

\usepackage{color}


\usepackage[ngerman]{babel}

\usepackage[utf 8]{inputenc}

\usepackage{float}

\usepackage{graphicx}

\usepackage{alltt}


\subject{Diskrete Strukturen}
\title{Blatt [NR]}
\author{Eike Janning, Tobias Nagel, Jan Paul Wessendorf}
\date{Abgabe: [TAG].[MONAT].2019}

\renewcommand{\arraystretch}{1.25}



\begin{document}
	\maketitle
	\hrulefill
	
	\section*{Aufgabe 7}
		\begin{itemize}
			\item[a)] Zu zeigen: $\forall x,y \in \mathbb{N}_0 \backslash \{ 0 \}$, x,y ungerade $: x*y$ ungerade\\
			Laut Definition von ungeraden Zahlen:\\
			$\exists m$ gerade : $x = 2*m+1$\\
			$\exists n$ gerade : $y = 2*n+1$
			\begin{center}
				$\Rightarrow x*y = (2*m+1)*(2*n+1)$\\
				$\Rightarrow x*y = 4mn + 2m + 2n + 1$
			\end{center}
			Da $4mn$, $2m$ und $2n$ gerade (Beweis siehe Nr 8a, 2 gerade, 4 gerade) sowie $1$ ungerade, ist $x*y$ ungerade.
			
			\item[b)] Zu zeigen: $\forall x \in \mathbb{R} : x^2 - 4x + 5 > 0$\\
			Mit binomischer Formel: $x^2 - 4x + 5 > 0 \Leftrightarrow (x + 2)^2 + 1 > 0$\\\\
			Da jede quadrierte Zahl $\ge 0$ ist und $1 > 0$ ist der Beweis trivial.
		\end{itemize}
	
	
	\section*{Aufgabe 9}
		\begin{itemize}
			\item[a)] Im Induktionsschritt wird angenommen dass $a=0$ und $b=0$ sind um den Induktionsanfang zu verwenden.
		\end{itemize}
\end{document}