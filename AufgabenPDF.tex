\documentclass[11pt]{article}
\usepackage[utf8]{inputenc}
\usepackage{geometry}
\geometry{a4paper}
\geometry{margin=0.6in}
\usepackage{graphicx}
% \usepackage[parfill]{parskip} % Activate to begin paragraphs with an empty line rather than an indent
\usepackage{booktabs}
\usepackage{array}
\usepackage{paralist}
\usepackage{verbatim}
\usepackage{subfig}
\usepackage{fancyhdr}
\usepackage{amsmath, amssymb}
\pagestyle{fancy}
\renewcommand{\headrulewidth}{0pt}
\lhead{}\chead{}\rhead{}
\lfoot{}\cfoot{\thepage}\rfoot{}
\usepackage{sectsty}
\allsectionsfont{\sffamily\mdseries\upshape}
\usepackage[nottoc,notlof,notlot]{tocbibind}
\usepackage[titles,subfigure]{tocloft}
\usepackage{stmaryrd}
\renewcommand{\cftsecfont}{\rmfamily\mdseries\upshape}
\renewcommand{\cftsecpagefont}{\rmfamily\mdseries\upshape}

\title{Diskrete Strukturen - Blatt 1}
\author{Eike Janning (458 610), Tobias Nagel (459 516), Jan Wessendorf (459 620)}
\date{Abgabe: 19.04.2019}





\begin{document}
\maketitle

\section*{Aufgabe 1}

\pagebreak

\section*{Aufgabe 2}
	 Geben Sie eine induktive Definition für arithmetische Ausdrücke mit Multiplikation und Addition (einschließlich Klammersetzung) für die natürlichen Zahlen an.\\
	 	\begin{align*}
		 	a, b_n \in \mathbb{N}, a = 0, b_1 = 1, b_n = n*(a+b_1)
	 	\end{align*}\\\\\\

\section*{Aufgabe 3}
	
\pagebreak

\section*{Aufgabe 4}
Sei $m_i$ eine Menge, so geben Sie jeweils die Elemente der Mengen M3, M17 und M42 an.\\\\
	1. $ i \in M$\\
	2. $m \in M \land m\: \textit{\textbf{ungerade}} \Rightarrow (3m+1) \in M$\\
	3. $m \in M \land m\: \textit{\textbf{gerade}} \Rightarrow (\frac{m}{2}) \in M$
		\begin{align*}
			M_3 &=: \left\{1,2,3,4,5,8,10,16\right\}\\
			M_{17} &=: \left\{1,2,4,5,8,10,13,16,17,20,26,40,52\right\}\\
			M_{42} &=: \left\{1,2,4,8,16,21,32,42,64\right\}
		\end{align*}\\\\\\
		
		
\section*{Aufgabe 5}
	 Geben Sie zu jeder der nachfolgenden Aussagen an, ob sie korrekt ist.
		\begin{align*}
			\varnothing &= \varnothing \quad \textit{Wahr}\\
			\varnothing &\subseteq \varnothing \quad \textit{Wahr}\\
			\varnothing &\subsetneq \varnothing \quad \textit{Falsch}\\
			\varnothing &\in \varnothing \quad \textit{Falsch}\\
			\varnothing &\subseteq \left\{\varnothing\right\} \quad \textit{Wahr}\\
			\varnothing &\subsetneq \left\{\varnothing\right\} \quad \textit{??}\\
		\end{align*}\\\\

\section*{Aufgabe 6}

\subsection*{a)}
$(A \cap B^C)^C$ (Regeln von de Morgan)\\
$= A^C \cup (B^C)^C$ (Involution)\\
$= A^C \cup B$\\

\subsection*{b)}
$(((A\cap B) \cup C)^C \cap B^C)^C$ (Regeln von de Morgan)\\
$= (((A\cap B)\cup C)^C)^C \cup (B^C)^C$ (Involution)\\
$= ((A\cap B)\cup C) \cup B$ (Assoziativgesetz)\\
$= (A\cap B) \cup (C \cup B)$ (Kommutativgesetz)\\
$= (B\cap A) \cup (B \cup C) $ (Assoziativgesetz)\\
$= (B\cap A)\cup B) \cup C$  (Kommutativgesetz)\\
$= (B\cup (B\cap A))\cup C$ (Absorptionsgesetz)\\
$= B \cup A$



\pagebreak



\end{document}
